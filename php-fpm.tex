\documentclass[11pt]{exam}
\usepackage[T1]{fontenc}
\usepackage[utf8]{inputenc}
\usepackage[scale=0.8]{geometry}
\usepackage[spanish]{babel}
\usepackage{amssymb}
\usepackage{float}
\usepackage{graphicx,xcolor}
\usepackage{hyperref}
\title{Configuración de nginx con php-fpm} \author{Maximiliano Osorio \\ mosorio@inf.utfsm.cl}
\date{Valparaíso, \today}
\usepackage{hyperref}
\hypersetup{
    colorlinks = true,
}
\usepackage{listings}

\lstset{
    frame=single,
    breaklines=true,
    postbreak=\raisebox{0ex}[0ex][0ex]{\ensuremath{\color{red}\hookrightarrow\space}}
}
\usepackage{marginnote}

\begin{document}
\maketitle
\makebox[\textwidth]{Nombre:\enspace\hrulefill}
\vspace{5mm}

Nota:
\begin{itemize}
	\item Los textos de documentación o manuales serán presentados en su idioma original.
	\item Ante cualquier duda, avise.
\end{itemize}

\section{Introducción}
PHP-FPM (FastCGI Process Manager) is an alternative PHP FastCGI implementation with some additional features useful for sites of any size, especially busier sites.

\section{Ejercicio}
En este ejercicio usted configurará la web de Homero Simpsons para servir contenido dinámico escrito en \texttt{php}. Los requerimientos de Homero son: Utilizar Nginx sí o sí. Dado eso, usted configurará Nginx y php-fpm para cumplir con el requerimiento.

\subsection{Configuración php-fpm}
\begin{enumerate}
	\item Realice la instalación del paquete \texttt{php-fpm}
	\item ¿Que versión de php se encuentra instalando?. Comente sobre la versión de php: ¿es nueva? ¿es segura?.
	\vspace{0.9in}
	\item Identifique los archivos de configuración de php-fpm, escriba el comando que utilizó.
	\vspace{0.9in}
	\item php-fpm puede crear y administrar múltiples pools de procesos php, también conocidos como workers que reciben y sirven requests para ejecutar archivos php del directorio web. Ahora php-fpm puede correr distintos pools, por ejemplo, se puede utilizar distintas configuraciones de pool para distintas aplicaciones. En \texttt{/etc/php-fpm.d/} se encuentran las configuraciones. Considerando lo anterior, describa las siguientes variables.
	\begin{enumerate}
		\item listen: Explique la definición y la diferencia entre usar un unix socket o un TCP socket
		\item user y group. Explique la definición y que utilizaría en su servidor actual.
		\item pm: Explique la definición y que valor usaría.
		\item pm.max\_children, pm.start\_servers, pm.min\_spare\_servers, pm.max\_spare\_servers y pm.max\_requests. Explique la definición de cada una de las variables y que aspectos consideraría para calcular sus valores.
		\item slowlog: Explique la definición.
	\end{enumerate}
	\vspace{5in}
	\item Inicie y active el servicio.
	\item Verifique que el proceso se encuentra corriendo. ¿Cuántos proceso se encuentra corriendo? Justifique el número.
\end{enumerate}

\subsection{Configuración nginx para php-fpm}
En esta sección usted modificará el archivo de configuración del virtual host (server block) de la web de Homero, según la experiencia anterior se encuentra en \texttt{/etc/nginx/conf.d/homerswebpage.conf}
\begin{enumerate}
	\item Primero lea el articulo de DigitalOcean \footnote{Digital Ocean normalmente tiene buenos árticulos} en la sección \textit{Matching Location Blocks}.
	\item Explique cómo funciona los matchs:
	\begin{itemize}
		\item (none)
		\item =
		\item $\thicksim$
		\item $\thicksim$*
		\item $\wedge\thicksim$
	\end{itemize}
	\newpage
	\item Explique: try\_files, fastcgi\_pass, fastcgi\_param
	\begin{lstlisting}
	...
    index index.php index.html index.htm;
    root /var/www/homerswebpage/;
    location / {
        try_files $uri $uri/ =404;
    }
    error_page 404 /404.html;
    error_page 500 502 503 504 /50x.html;
    location = /50x.html {
        root /usr/share/nginx/html;
    }

    location ~ \.php$ {
        try_files $uri =404;
        fastcgi_pass unix:/var/run/php-fpm/php-fpm.sock;
        fastcgi_index index.php;
        fastcgi_param SCRIPT_FILENAME $document_root$fastcgi_script_name;
        include fastcgi_params;
    }
	\end{lstlisting}
	\vspace{3.5in}
	\item Configure el virtual host del homero con lo anterior.
\end{enumerate}

\end{document}